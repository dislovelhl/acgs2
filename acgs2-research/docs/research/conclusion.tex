\clearpage
\section{Conclusion}\label{sec:conclusion}
This paper has introduced ACGS-2, a production-ready technical infrastructure
for constitutional AI governance that provides reliable foundations upon which
human-centered governance processes can operate. Through our comprehensive
evaluation, we have demonstrated both the capabilities and fundamental
limitations of technical approaches to this sociotechnical challenge. Our
analysis shows that while sophisticated engineering can achieve exceptional
performance (\throughput{}, \pnnlatency{} $P_{99}$ latency) and perfect
compliance across \testoperations{} test operations, these achievements do not
directly translate to democratic legitimacy and raise important questions about
the role of ultra-fast decision making in democratic processes. \sloppy \fussy

Our primary contribution is a methodological framework for evaluating
constitutional AI systems based on their capacity to support, rather than
supplant, democratic deliberation. We argue that the main value of systems like
ACGS-2 lies in providing a stable, verifiable foundation for human-led
governance. The empirical results---\constitutionalcompliance{} compliance
across \testoperations{} comprehensive test operations, ultra-fast
(\pnnlatency{} $P_{99}$) validation with cryptographic hash verification
(\texttt{\small{cdd01ef066bc6cf2}}), and \uptime{} reliability---demonstrate
exceptional technical capabilities. However, these must be weighed against the
significant sociotechnical challenges: the \enquote{synthetic constitution}
problem, the implications of ultra-high-speed decision making, and the inherent
tension between automation speed and the pace of democratic deliberation.
\sloppy

This work represents a methodological contribution to constitutional AI
research by demonstrating how to evaluate systems based on their capacity to
support democratic processes rather than technical performance alone. By
providing both technical infrastructure and an honest analysis of its
limitations, we hope to advance the field's understanding of what automation
can and cannot accomplish in constitutional governance. The algorithmic
breakthrough opportunities we identify---adaptive constitutional
interpretation, decentralized democratic consensus, context-aware enforcement,
and real-time democratic deliberation---offer concrete pathways for future
research that could meaningfully enhance constitutional governance systems
while preserving democratic legitimacy. Our primary lesson is that technical
excellence in constitutional AI requires not just sophisticated engineering,
but deep engagement with the sociotechnical challenges of democratic
governance.

\subsection{Reproducibility and Open Science}
To support the research community and enable practical deployment, we commit to
releasing:
\begin{itemize}[leftmargin=*,itemsep=1pt,parsep=1pt]
    \item Complete system architecture documentation and deployment templates
    \item Constitutional validation framework with \testoperations{} comprehensive test
          operations
    \item Performance benchmarking tools and datasets
    \item Democratic governance integration patterns
    \item Security audit reports and best practices
\end{itemize}

\paragraph{Data and Code Availability.} All source code, documentation, and supplementary materials for this research
are available through Zenodo with DOI: \textbf{10.5281/zenodo.16415583}. This
permanent repository includes the complete ACGS-2 implementation, experimental
datasets, figure generation scripts, and reproducibility guidelines.
