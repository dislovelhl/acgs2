\appendix
\section{Supplementary Materials}\label{app:supplementary}
This appendix provides additional details regarding the experimental setup and
system configuration to ensure reproducibility.

\subsection{Glossary of Key Terms}\label{app:glossary}
\begin{description}[leftmargin=1em,style=nextline]
    \item[Constitutional AI (CAI)] A paradigm for aligning AI systems with human values
          by having them adhere to an explicit set of principles or a ''constitution.''
    \item[Constitutional Hash] A cryptographic identifier (cdd01ef066bc6cf2) that ensures
          policy consistency across distributed services through version-controlled
          verification.
    \item[Microservices Architecture] A software design pattern where applications are
          built as a collection of loosely coupled, independently deployable services.
    \item[P99 Latency] The response time below which 99\% of requests are served (e.g.,
          2.1ms means 99\% of requests complete in under 2.1 milliseconds).
    \item[RPS (Requests Per Second)] A measure of system throughput indicating how many
          requests the system can process per second.
    \item[Circuit Breaker] A design pattern that prevents cascading failures by
          temporarily blocking requests to failing services.
    \item[Blue-Green Deployment] A deployment strategy that reduces downtime by running
          two identical production environments (blue and green), switching traffic
          between them.
    \item[OPA (Open Policy Agent)] An open-source policy engine that provides a unified
          framework for policy enforcement across software systems.
    \item[Rego] The declarative query language used by Open Policy Agent for writing
          policies.
    \item[Chaos Engineering] The practice of intentionally introducing failures into a
          system to test its resilience and identify weaknesses.
    \item[Cache Hit Rate] The percentage of data requests that can be served from cache
          rather than requiring expensive database queries.
    \item[Byzantine Fault Tolerance] A property of distributed systems that can continue
          operating correctly even when some components fail or behave maliciously.
    \item[Formal Verification] Mathematical techniques used to prove that a system
          satisfies certain specifications or properties.
    \item[Service Mesh] An infrastructure layer that handles service-to-service
          communication, providing features like traffic management and security.
\end{description}

\subsection{Experimental Environment}
All performance and validation tests were conducted on a dedicated Kubernetes
cluster with the following specifications:
\begin{itemize}
    \item \textbf{Orchestration:} Kubernetes v1.28.2 running on 5 worker nodes.
    \item \textbf{Worker Nodes:} Each node was a cloud instance with 8 vCPUs, 32 GB RAM, and local NVMe SSD storage.
    \item \textbf{Networking:} Calico CNI with a 10 Gbps interconnect.
    \item \textbf{Service Mesh:} Istio v1.20 was used for traffic management, mTLS encryption, and observability, but was disabled during raw performance benchmarks to isolate the application's latency.
    \item \textbf{Database:} PostgreSQL 15.4, running on a dedicated high-I/O instance with 4 vCPUs and 16 GB RAM.
    \item \textbf{Cache:} Redis 7.2.1, running in a clustered configuration across 3 dedicated nodes.
    \item \textbf{Monitoring Stack:} Prometheus v2.47.0, Grafana v10.1.5, and AlertManager v0.26.0.
\end{itemize}

\subsection{Load Testing Configuration}
\begin{itemize}
    \item \textbf{Tool:} k6 v0.48.0.
    \item \textbf{Workload:} A script simulating 80\% read operations (policy evaluations) and 20\% write operations (actions requiring new audit log entries).
    \item \textbf{Traffic Pattern:} A ramp-up phase from 0 to 1,000 virtual users over 10 minutes, a sustained load phase of 20 minutes at 1,000 users, and a ramp-down phase.
    \item \textbf{Metrics:} P99 latency, requests per second (RPS), and HTTP success rate were collected by k6. System-side metrics (CPU, memory, cache hit rate) were collected by Prometheus.
\end{itemize}

\subsection{Formal Proofs and Verification Artifacts}\label{app:formal_proofs}
We provide summarized Z3 SMT solver outputs, proof obligations, and verification logs for the constitutional reasoning engine. Full proof artifacts are available in the accompanying supplementary materials to support reproducibility of the formal compliance claims.
