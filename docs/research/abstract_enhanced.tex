\begin{abstract}
We present ACGS-2, a prototype constitutional AI governance platform that treats constitutional AI as infrastructure for democratic processes rather than as an automated solution to constitutional design. The system implements a \microservices-service distributed architecture with cryptographically verified constitutional consistency via a constitutional hash, a transformer-based constitutional reasoning engine, and a multi-tier policy-as-code pipeline integrating OPA/Rego and Z3-based formal verification. Using synthetic governance scenarios reconstructed from patterns in constitutional and AI governance literature, we evaluate ACGS-2 in a controlled Kubernetes cluster. The platform achieves millisecond-scale P99 end-to-end latency (\pnnlatency), supports sustained request throughput (\throughputrange), maintains high cache hit rates (\cachehitrate) and constitutional compliance (\constitutionalcompliance), and provides continuous availability (\uptime) under development workloads.

Methodologically, we (1) formalize constitutional AI as a quadruple $\mathcal{C} = (\mathcal{P}, \mathcal{R}, \mathcal{E}, \mathcal{V})$ relating principles, reasoning, enforcement, and verification; (2) introduce Democratic Facilitation Capacity (DFC) as a composite metric for assessing how technical systems preserve deliberation, support stakeholder engagement, enable constitutional evolution, and provide transparency; and (3) characterize the ``synthetic constitution problem,'' highlighting the gap between performance on synthetic constitutional frameworks and effectiveness under authentic, democratically derived constitutions.

All experiments rely on synthetic data and simulated stakeholders; no real institutions or citizens are involved. DFC weights are heuristically chosen and require empirical validation. We position ACGS-2 as a reproducible research artifact and as a case study in the laboratory-to-production gap for constitutional AI, arguing that future work must focus on integrating authentic stakeholders, multi-jurisdictional constitutional frameworks, and longitudinal democratic evaluation.
\end{abstract}

\begin{CCSXML}
<ccs2012>
<concept>
<concept_id>10003456.10003462.10003588.10003589</concept_id>
<concept_desc>Social and professional topics~AI governance</concept_desc>
<concept_significance>500</concept_significance>
</concept>
<concept>
<concept_id>10010147.10010178.10010179.10010182</concept_id>
<concept_desc>Computing methodologies~Distributed computing methodologies</concept_desc>
<concept_significance>500</concept_significance>
</concept>
<concept>
<concept_id>10002978.10003022.10003023</concept_id>
<concept_desc>Security and privacy~Formal methods</concept_desc>
<concept_significance>300</concept_significance>
</concept>
<concept>
<concept_id>10003456.10003462.10003590</concept_id>
<concept_desc>Social and professional topics~Computing / technology policy</concept_desc>
<concept_significance>400</concept_significance>
</concept>
<concept>
<concept_id>10010147.10010257.10010293.10010294</concept_id>
<concept_desc>Computing methodologies~Neural networks</concept_desc>
<concept_significance>400</concept_significance>
</concept>
</ccs2012>
\end{CCSXML}

\ccsdesc[500]{Social and professional topics~AI governance}
\ccsdesc[500]{Computing methodologies~Distributed computing methodologies}
\ccsdesc[400]{Social and professional topics~Computing / technology policy}
\ccsdesc[400]{Computing methodologies~Neural networks}
\ccsdesc[300]{Security and privacy~Formal methods}

\keywords{Constitutional AI, AI Governance, Distributed Systems, Democratic Facilitation Capacity, Formal Verification, Policy-as-Code, Synthetic Constitution Problem}
